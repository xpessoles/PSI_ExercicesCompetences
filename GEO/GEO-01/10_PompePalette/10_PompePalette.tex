\normalfalse \difficiletrue \tdifficilefalse
\correctiontrue

%\UPSTIidClasse{11} % 11 sup, 12 spé
%\newcommand{\UPSTIidClasse}{12}

\exer{Pompe à palettes  $\star\star$ \label{CIN:01:B2:12:10}}
\setcounter{question}{0}
\marginnote{\xpComp{GEO}{01}}
%\marginnote{\xpComp{GE0}{01}}
%\UPSTIcompetence{B2-12}
\index{Compétence B2-12}
\index{Compétence CIN-01}
\index{Compétence GEO-01}
\index{Pompe à palettes}
\ifcorrection
\else
\marginnote{\textbf{Pas de corrigé pour cet exercice.}}
\fi

\ifprof
\else
Soit le mécanisme suivant. On a $\vect{AO}=e\vect{i_0}$ et $\vect{AB}=\lambda(t)\vect{i_1}$. De plus $e=\SI{10}{mm}$ et $R=\SI{20}{mm}$. Le contact entre \textbf{0} et \textbf{2} en $B$ est maintenu en permanence (notamment par effet centrifuge lors de la rotation de la pompe).
\begin{marginfigure}
\includegraphics[width=\linewidth]{10_01}
\end{marginfigure}
\fi


\question{Tracer le graphe des liaisons.}
\ifprof
\begin{marginfigure}
\includegraphics[width=5cm]{10_01_cor}
\end{marginfigure}
\else
\fi


\question{Retracer le schéma cinématique pour $\theta(t)=0 \,\text{rad}$.}
\ifprof
\begin{marginfigure}
\includegraphics[width=5cm]{10_02_cor}
\end{marginfigure}
\else
\fi

\question{Retracer le schéma cinématique pour $\theta(t)=\pi\,\text{rad}$.}
\ifprof
\begin{marginfigure}
\includegraphics[width=5cm]{10_03_cor}
\end{marginfigure}

\else
\fi


\question{En déduire la course de la pièce \textbf{2}.}
\ifprof
La course de la pièce 2 est donnée par la différence entre la longueur $AB$ maximale et $AB$ minimale : $c= 30-10 = \SI{20}{mm}$.
\else
\fi



\ifprof
\else

\marginnote{Corrigé  voir \ref{CIN:01:B2:12:10}.}

\fi