\normaltrue
\correctiontrue

%\UPSTIidClasse{11} % 11 sup, 12 spé
%\newcommand{\UPSTIidClasse}{12}


\exer{Mouvement TT -- $\star$ \label{CIN:01:B2:12:03}}
\setcounter{question}{0}\marginnote{\xpComp{CIN}{01}}%\UPSTIcompetence{B2-12}
\index{Compétence B2-12}\index{Compétence CIN-01}
\index{Mécanisme à 2 translations}
\ifcorrection
\else
\marginnote{\textbf{Pas de corrigé pour cet exercice.}}
\fi

\ifprof
\else
Soit le mécanisme suivant. On note $\vect{AB}=\lambda(t)\vect{i_0}$ et $\vect{BC}=\mu(t)\vect{j_0}$.
\begin{marginfigure}
\includegraphics[width=\linewidth]{03_TT_01}
\end{marginfigure}
\fi



\question{Tracer le graphe des liaisons.}
\ifprof
\begin{marginfigure}
\includegraphics[width=\linewidth]{03_TT_01_c}
\end{marginfigure}
\else
\fi

\question{Retracer le schéma cinématique pour $\lambda=\SI{10}{mm}$ et $\mu=\SI{10}{mm}$.}
\ifprof
\begin{marginfigure}
\includegraphics[width=\linewidth]{03_TT_02_c}
\end{marginfigure}
\else
\fi

\question{Retracer le schéma cinématique pour $\lambda=\SI{20}{mm}$ et $\mu=\SI{10}{mm}$.}
\ifprof
\begin{marginfigure}
\includegraphics[width=.8\linewidth]{03_TT_03_c}
\end{marginfigure}
\else
\fi


\ifprof
\else

\marginnote{Corrigé  voir \ref{CIN:01:B2:12:03}.}

\fi


