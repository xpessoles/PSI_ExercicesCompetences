\normaltrue
\correctionfalse

%\UPSTIidClasse{11} % 11 sup, 12 spé
%\newcommand{\UPSTIidClasse}{12}

\exer{Mouvement RT  $\star$ \label{C2:07:05}}
\setcounter{question}{0}\UPSTIcompetence[2]{B2-14}
\UPSTIcompetence[2]{B2-15}
\UPSTIcompetence[2]{C2-07}
\index{Compétence B2-14}
\index{Compétence B2-15}
\index{Compétence C2-07}
\index{Torseur des actions mécaniques transmissibles}
\index{Torseur d’une action mécanique extérieure}
\index{Principe fondamental de la statique}
\index{PFS}
\index{Mécanisme à 1 rotation et 1 translation}
\ifcorrection
\else
\marginnote{\textbf{Pas de corrigé pour cet exercice.}}
\fi

\ifprof
\else
Soit le mécanisme suivant. On a $\vect{AB}=\lambda(t)\vect{i_1}$. De plus :
\begin{itemize}
\item $G_1$ désigne le centre d'inertie de \textbf{1} et $\vect{AG_1}=L_1\vect{i_1}$, on note $m_1$ la masse de \textbf{1}; %et $\inertie{G_1}{1}=\matinertie{A_1}{B_1}{C_1}{0}{0}{0}{\bas{1}}$; 
\item $G_2=B$ désigne le centre d'inertie de \textbf{2}, on note $m_2$ la masse de \textbf{2}.% et $\inertie{G_2}{2}=\matinertie{A_2}{B_2}{C_2}{0}{0}{0}{\bas{2}}$.
\end{itemize}


Un moteur électrique positionné entre \textbf{0} et \textbf{1} permet de maintenir \textbf{1} en équilibre.
Un vérin électrique positionné entre \textbf{1} et \textbf{2} permet de maintenir \textbf{2} en équilibre.

L'accélération de la pesanteur est donnée par $\vect{g}=-g\vect{j_0}$.

\begin{center}
\includegraphics[width=\linewidth]{05_RT_01}
\end{center}
\fi

\question{Réaliser le graphe d'analyse en faisant apparaître l'ensemble des actions mécaniques.}
\ifprof
\else
\fi


\question{Donner le couple moteur et l'effort à fournir par le vérin pour maintenir le système à l'équilibre.}
\ifprof
\else
\fi

\question{Donner les actions mécaniques dans chacune des liaisons.}
\ifprof
\else
\fi

\ifprof
\else
\begin{flushright}
\footnotesize{Corrigé  voir \ref{C2:07:05}.}
\end{flushright}%
\fi