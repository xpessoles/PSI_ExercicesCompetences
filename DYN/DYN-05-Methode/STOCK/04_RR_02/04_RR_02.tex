\normaltrue
\correctionfalse

%\UPSTIidClasse{11} % 11 sup, 12 spé
%\newcommand{\UPSTIidClasse}{12}

\exer{Mouvement RR  $\star$ \label{DYN:05:B2:13:04:02}}
\setcounter{question}{0}\marginnote{\xpComp{DYN}{05}}%\UPSTIcompetence{B2-13}
\index{Compétence B2-13}\index{Compétence DYN-05}
\index{Mécanisme à 2 rotations}
\ifcorrection
\else
\marginnote{\textbf{Pas de corrigé pour cet exercice.}}
\fi

\ifprof
\else
Soit le mécanisme suivant. On a $\vect{AB}=R\vect{i_1}$ avec $R=\SI{20}{mm}$ et  
$\vect{BC}=L\vect{i_1}$ avec $L=\SI{15}{mm}$.
\begin{marginfigure}
\includegraphics[width=\linewidth]{04_RR_01}
\end{marginfigure}
\fi

\question{Déterminer $\vectv{C}{2}{0}$ par dérivation vectorielle ou par composition.}
\ifprof
\else
\fi

\question{Donner le torseur cinématique $\torseurcin{V}{2}{0}$ au point $C$.}
\ifprof
\else
\fi

\question{Déterminer $\vectg{C}{2}{0}$.}
\ifprof
\else
\fi

\ifprof
\else

\marginnote{Corrigé voir \ref{DYN:05:B2:13:04:02}.}

\fi