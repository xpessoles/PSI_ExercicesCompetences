\normaltrue \difficilefalse \tdifficilefalse
\correctionfalse

%\UPSTIidClasse{11} % 11 sup, 12 spé
%\newcommand{\UPSTIidClasse}{12}

\exer{Pale hélicoptère$\star$ \label{DYN:03:1026}}
\setcounter{question}{0}
\marginnote{\xpComp{DYN}{03}}%\UPSTIcompetence[2]{B2-10}
\index{Compétence B2-10}
\index{Compétence DYN-03}

\index{Matrice d'inertie}
\index{Caractéristiques inertielles}
\ifcorrection
\else
\marginnote{\textbf{Pas de corrigé pour cet exercice.}}
\fi


\ifprof
\else
\textbf{Les notations sont indépendantes des parties précédentes. %Aucun calcul intégral n'est attendu dans cette partie.
}

La figure \ref{fig:2005_EA_fig_09} illustre le paramétrage d'une pale d'hélicoptère (partielle). On note \textbf{1} la partie qui fit le raccord avec le rotor et \textbf{2} la pale. La pièce 2 est pleine.


\begin{figure}[!h]
\centering
\includegraphics[width=.7\linewidth]{2005_EA_fig_09} % remplacer par votre fichier
    \caption{Paramétrage d'une pale}
    \label{fig:2005_EA_fig_09}
\end{figure}
\fi

\question{Déterminer la position du centre d'inertie $G_1$ de \textbf{1} dans le repère $\repere{O}{x}{y}{z}$.}
\ifprof
\begin{corrige}
La portion 1 est un cylindre $\vect{OG_1} = -\dfrac{\ell}{2}\vx{}$.
\end{corrige}
\else
\fi

\question{Déterminer la masse $m_1$ du solide \textbf{1} et la masse $m_2$ du solide \textbf{2}.  On fera l'hypothèse que ce sont tous deux des solides homogènes de masse volumique $\mu$.}
\ifprof
\begin{corrige}
$m_1 =\mu \pi r^2 \ell$

$m_2 = \mu \dfrac{1}{2} \pi R^2 L + \mu RHL$
\end{corrige}
\else
\fi

\question{Déterminer la position du centre d'inertie $G_2$ de coordonnées $(a_2,b_2,c_2)$ de \textbf{2} dans le repère $\repere{O}{x}{y}{z}$.}
\ifprof
\begin{corrige}
La portion $2_a$ est composée d'un demi cylindre de centre d'inertie $G_a$. Pour des raisons de symétrie, seule la composante suivant $\vz{}$ est à déterminer.

On a donc $m_a \vect{OG_a}\cdot \vz{} = \iint \vect{OP}\cdot \vz{} \d m$ 
$ =\iint \rho \sin \theta \mu      \rho \d\rho \d \theta \d z$  
$ = - \mu    \dfrac{1}{3}R^3  L \left[ \cos \theta \right]_{0}^{\pi}  $  
$ =  2 \mu    \dfrac{1}{3}R^3  L $

Avec $m_a = \mu \pi R^2/2L$, on a $ \vect{OG_a}\cdot \vz{} = \dfrac{4}{3\pi}R  $.

On a donc $\vect{OG_a}=-\left(\ell+\dfrac{L}{2}\right) \vx{}+\dfrac{4}{3\pi}R \vz{}$.

La portion $2_b$ es un prisme. On a $\vect{OG_b}=-\left(\ell+\dfrac{L}{2}\right) \vx{}-\dfrac{1}{3}H \vz{}$. Elle est de masse $m_b = \mu RHL$.


On a alors $\vect{OG_2} = \dfrac{m_a \vect{OG_a} + m_b \vect{OG_b} }{m_2}$
\end{corrige}
\else
\fi



\question{Donner la position du centre d'inertie $G$ de l'ensemble \textbf{1+2} dans le repère $\repere{O}{x}{y}{z}$.}
\ifprof
\begin{corrige}
$\vect{OG} = \dfrac{m_1 \vect{OG_1} + m_2 \vect{OG_2} }{m_1+m_2} =-a\vx{}-c\vz{}$ .
\end{corrige}
\else
\fi


On note $\inertie{P}{i} = \matinertie{A_i}{B_i}{C_i}{-D_i}{-E_i}{-F_i}{\rep{}}$  la matrice d'inertie du solide $i$ au point $P$.

\question{Donner, \textbf{en justifiant} la forme de $\inertie{O}{1}$, $\inertie{G_2}{2}$, $\inertie{O}{1+2}$. }
\ifprof
\begin{corrige}~\\

En $O$ il y a une inifinité de plans de symétrie pour 1. De plus, $\vy{}$ et $\vz{}$ jouent le même rôle. 
$\inertie{O}{1} = \matinertie{A_1}{B_1}{B_1}{0}{0}{0}{\rep{}}$.

En $G_2$ il y a 2 plans de symétrie perpendiculaires pour 2 $(G_2\,\vy{},\vz{})$ et $(G_2\,\vz{},\vx{})$ .
$\inertie{G_2}{2} = \matinertie{A_2}{B_2}{C_2}{0}{0}{0}{\rep{}}$.


En $O$ il y a 1 plans de symétrie permendiculaires pour 1+2  $(G_2\,\vz{},\vx{})$ .
$\inertie{O}{1+2} = \matinertie{A_{1+2}}{B_{1+2}}{C_{1+2}}{0}{-E_{1+2}}{0}{\rep{}}$.

\end{corrige}
\else
\fi


\question{Déterminer $\inertie{O}{1+2}$ en fonction des composantes des matrices $\inertie{O}{1}$ et $\inertie{G_2}{2}$ et des grandeurs que vous jugerez utiles. On notera $\vect{OG_2} = a\vx{}+b\vy{}+c\vz{}$ avec $(a,b,c)\in \mathbb{R}^3$ (on pourra éventuellement simplifier l'expression de $\vect{OG_2}$ en fonction des raisonnements précédents).}
\ifprof
\begin{corrige}

On a  $\vect{OG_2} = a\vx{}+c\vz{}$ et 
$\inertie{O}{2} = \inertie{G_2}{2} + \matinertie{m_2c^2}{m_2 \left(a^2+c^2\right)}{m_2a^2}{0}{-m_2 ac }{0}{}$

$=\matinertie{A_2+m_2c^2}{B_2 + m_2 \left(a^2+c^2\right)}{C_2 + m_2a^2}{0}{-m_2 ac }{0}{}$



$\inertie{O}{1+2} =\matinertie{A_1 + A_2+m_2c^2}{B_1 + B_2 + m_2 \left(a^2+c^2\right)}{C_1 + C_2 + m_2a^2}{0}{-m_2 ac }{0}{}$
\end{corrige}
\else
\fi



\ifprof
\else
\marginnote{
\begin{solution}
%\begin{enumerate}
%\item $\inertie{O}{1}=\matinertie{A_1}{B_1}{C_1}{0}{0}{0}{\base{x_1}{y_1}{z_0}}$.
%\item $C_1 = \dfrac{m_1}{12}\left(b^2+c^2\right)$.
%\item $\inertie{G_2}{1}=\matinertie{A_2}{A_2}{C_2}{0}{0}{0}{\base{x_1}{y_1}{z_0}}$.
%\item $C'_2 =m_2\dfrac{r^2}{2}$.
%\item $C_2 = m_2 \left(\dfrac{r^2}{2} +a^2 \right)$.
%\end{enumerate}
\end{solution}}


\marginnote{Corrigé voir \ref{DYN:03:B2:1026}.}

\fi