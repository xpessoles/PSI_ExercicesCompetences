\normaltrue \difficilefalse \tdifficilefalse
\correctionfalse

%\UPSTIidClasse{11} % 11 sup, 12 spé
%\newcommand{\UPSTIidClasse}{12}

\exer{Pompe à palettes  $\star$ \label{B2:13:10}}
\setcounter{question}{0}\UPSTIcompetence[2]{B2-13}
\index{Compétence B2-13}
\index{Pompe à palettes}
\ifcorrection
\else
\marginnote{\textbf{Pas de corrigé pour cet exercice.}}
\fi

\ifprof
\else
Soit le mécanisme suivant. On a $\vect{AO}=e\vect{i_0}$ et $\vect{AB}=\lambda(t)\vect{i_1}$. De plus $e=\SI{10}{mm}$ et $R=\SI{20}{mm}$. Le contact entre \textbf{0} et \textbf{2} en $B$ est maintenu en permanence (notamment par effet centrifuge lors de la rotation de la pompe).
\begin{center}
\includegraphics[width=\linewidth]{10_01}
\end{center}
\fi

Il est possible de mettre la loi entrée-sortie sous la forme *** (voir exercice \ref{C2:06:10}).

\question{Donner le torseur cinématique $\torseurcin{V}{2}{0}$ au point $B$.}
\ifprof
\else
\fi

\question{Déterminer $\vectg{B}{2}{0}$.}
\ifprof
\else
\fi


\ifprof
\else
\begin{flushright}
\footnotesize{Corrigé  voir \ref{B2:13:10}.}
\end{flushright}%
\fi