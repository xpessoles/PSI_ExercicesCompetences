\normalfalse \difficilefalse \tdifficiletrue
\correctionfalse

%\UPSTIidClasse{11} % 11 sup, 12 spé
%\newcommand{\UPSTIidClasse}{12}

\exer{Maxpid $\star\star\star$ \label{DYN:04:B2:13:18}}
\setcounter{question}{0}\UPSTIcompetence{B2-13}
\index{Compétence B2-13}\index{Compétence DYN-04}
\index{Maxpid}
\ifcorrection
\else
\marginnote{\textbf{Pas de corrigé pour cet exercice.}}
\fi

\ifprof
\else

Soit le schéma suivant. 
\begin{marginfigure}
\includegraphics[width=\linewidth]{18_01}
\end{marginfigure}
\fi

Par ailleurs $a=\SI{107,1}{mm}$, $b=\SI{80}{mm}$, $c=\SI{70}{mm}$, $d=\SI{80}{mm}$. Le pas de la vis est de $\SI{4}{mm}$.


Il est possible de mettre la loi entrée-sortie sous la forme *** (voir exercice \ref{C2:06:18}).

On définit le point $G$ tel que $\vect{OG}=L\vect{x_4}$.

\question{Donner le torseur cinématique $\torseurcin{V}{4}{0}$ au point $G$.}
\ifprof
\else
\fi

\question{Déterminer $\vectg{G}{4}{0}$.}
\ifprof
\else
\fi


\ifprof
\else

\marginnote{Corrigé voir \ref{DYN:04:B2:13:18}.}

\fi