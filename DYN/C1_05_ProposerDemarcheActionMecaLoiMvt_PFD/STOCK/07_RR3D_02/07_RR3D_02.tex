\normaltrue \difficilefalse \tdifficilefalse
\correctionfalse

%\UPSTIidClasse{11} % 11 sup, 12 spé
%\newcommand{\UPSTIidClasse}{12}

\exer{Mouvement RR 3D  $\star$ \label{B2:13:07:02}}
\setcounter{question}{0}\UPSTIcompetence{B2-13}
\index{Compétence B2-13}
\index{Mécanisme à 2 rotations 3D}
\ifcorrection
\else
\marginnote{\textbf{Pas de corrigé pour cet exercice.}}
\fi

\ifprof
\else
Soit le mécanisme suivant. On a $\vect{AB}=R\vect{i_1}$ et $\vect{BC}=\ell\vect{i_2}+r\vect{j_2}$. On note $R+\ell=L = \SI{20}{mm}$ et $r=\SI{10}{mm}$.
\begin{center}
\includegraphics[width=\linewidth]{07_RR3D_01}
\end{center}
\fi

\question{Déterminer $\vectv{C}{2}{0}$ par dérivation vectorielle ou par composition.}
\ifprof
\else
\fi

\question{Donner le torseur cinématique $\torseurcin{V}{2}{0}$ au point $C$.}
\ifprof
\else
\fi

\question{Déterminer $\vectg{C}{2}{0}$.}
\ifprof
\else
\fi


\ifprof
\else
\begin{flushright}
\footnotesize{Corrigé  voir \ref{B2:13:07}.}
\end{flushright}%
\fi