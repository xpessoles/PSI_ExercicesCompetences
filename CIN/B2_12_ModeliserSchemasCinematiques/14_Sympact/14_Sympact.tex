\normalfalse \difficiletrue \tdifficilefalse
\correctiontrue

%\UPSTIidClasse{11} % 11 sup, 12 spé
%\newcommand{\UPSTIidClasse}{12}

\exer{Barrière Sympact $\star\star$ \label{B2:12:14}}
\setcounter{question}{0}\UPSTIcompetence{B2-12}
\index{Compétence B2-12}
\index{Barrière Sympact}
\ifcorrection
\else
\marginnote{\textbf{Pas de corrigé pour cet exercice.}}
\fi

\ifprof
\else
Soit le mécanisme suivant. On a $\vect{AC}=H\vect{j_0}$ et $\vect{CB}=R\vect{i_1}$. De plus, 
$H=\SI{120}{mm}$ et $R=\SI{40}{mm}$. 

\begin{center}
\includegraphics[width=\linewidth]{14_01}
\end{center}
\fi


\question{Tracer le graphe des liaisons.}
\ifprof
\begin{center}
\includegraphics[width=.4\linewidth]{14_02_c}
\end{center}
\else
\fi

\question{Retracer le schéma cinématique pour $\theta(t)=\dfrac{\pi}{2}\,\text{rad}$.}
\ifprof
\begin{center}
\includegraphics[width=.5\linewidth]{14_01_c}
\end{center}

\else
\fi

\question{Retracer le schéma cinématique pour $\theta(t)=75\degres$.}
\ifprof
\else
\fi


\question{Dans l'hypothèse où la pièce \textbf{1} peut faire des tours complets, quelle doit être la longueur minimale de la pièce \textbf{2}.}
\ifprof
Dans cas, dans le pire des cas, $A$, $B$ et $C$ sont alignés (avec $B$ au-dessus de $C$). Il faut donc $AB = AC+CB = \SI{160}{mm}$.
\else
\fi

\question{Dans l'hypothèse où la pièce \textbf{2} fait \SI{12}{cm}, quel sera le débattement maximal de la pièce \textbf{1}.}
\ifprof
Comme je suis paresseux, j'ai réalisé la construction avec geogebra. On mesure \SI{160,8}{\degres}.
\begin{center}
\rotatebox{90}{\includegraphics[width=.35\linewidth]{14_03_c}}
\end{center}
\else
\fi



\ifprof
\else
\ifcolle
\else
\footnotesize
\begin{center}
\begin{tabular}{|p{.9\linewidth}|}
\hline
Indications :
\begin{enumerate}
\item .
\item .
\item .
\item \SI{160}{mm}.
\item \SI{160,8}{\degres}.
\end{enumerate} \\ \hline
\end{tabular}
\end{center}
\fi
\normalsize

\begin{flushright}
\footnotesize{Corrigé  voir \ref{B2:12:14}.}
\end{flushright}%
\fi