\normalfalse \difficiletrue \tdifficilefalse
\correctionfalse

%\UPSTIidClasse{11} % 11 sup, 12 spé
%\newcommand{\UPSTIidClasse}{12}

\exer{Mouvement RR 3D  $\star\star$ \label{B2:12:07}}
\setcounter{question}{0}\UPSTIcompetence{B2-12}
\index{Compétence B2-12}
\index{Mécanisme à 2 rotations 3D}
\ifcorrection
\else
\marginnote{\textbf{Pas de corrigé pour cet exercice.}}
\fi

\ifprof
\else
Soit le mécanisme suivant. On a $\vect{AB}=R\vect{i_1}$ et $\vect{BC}=\ell\vect{i_2}+r\vect{j_2}$. On note $R+\ell=L = \SI{20}{mm}$ et $r=\SI{10}{mm}$.
\begin{center}
\includegraphics[width=\linewidth]{07_RR3D_01}
\end{center}
\fi


\question{Tracer le graphe des liaisons.}
\ifprof
\else
\fi

\question{Retracer le schéma cinématique en 3D pour $\theta(t)=\dfrac{\pi}{2}\,\text{rad}$ et $\varphi(t)=\dfrac{\pi}{2}\, \text{rad}$.}
\ifprof
\else
\fi





\ifprof
\else
\begin{flushright}
\footnotesize{Corrigé  voir \ref{B2:12:07}.}
\end{flushright}%
\fi