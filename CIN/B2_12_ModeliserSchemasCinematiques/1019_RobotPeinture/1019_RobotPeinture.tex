\normalfalse \difficiletrue \tdifficilefalse
\correctionfalse
%\UPSTIidClasse{11} % 11 sup, 12 spé
%\newcommand{\UPSTIidClasse}{12}

\exer{Robot de toit  $\star\star$ \label{B2:12:1019}}
\setcounter{question}{0}\UPSTIcompetence{B2-12}
\index{Compétence B2-12}
\index{Schéma cinématique}

\ifcorrection
\else
\marginnote{\textbf{Pas de corrigé pour cet exercice.}}
\fi

\ifprof
\else
Soit le mécanisme donné au verso.
\fi


\ifprof
\else
La nomenclature est la suivante. 
\begin{multicols}{2}
\begin{center}
\begin{tabular}{|l|l|l|}
\hline
Rep & Nb  & Désignation \\ \hline \hline % & Matière \\ \hline \hline
1 & 1 & Carter inférieur fixe  \\ \hline %&  Al Si 13 \\ \hline 
2&
1&
Carter supérieur pivotant\\ \hline %& Al Si 13 \\ \hline 
3&
2 &
Ecrou hexagonal ISO 4032 - M10 \\ \hline %& \\ \hline 
4&
1&
Rondelle plate ISO 10673 – Type N - 10 \\ \hline %& \\ \hline 
5&
1&
Axe fileté à tête fendu \\ \hline %& \\ \hline 
6&
1&
Plat de fermeture%&
\\ \hline %S 235 \\ \hline 
7&
7&
Rondelle plate ISO 10673 -- Type N - 5 \\ \hline %& \\ \hline 
8&
1&
Bride de liaison support coussinets \\ \hline % &Al Cu 4 Mg Si \\ \hline 
9&
1&
Bride de liaison gauche \\ \hline % & Al Cu 4 Mg Si   \\ \hline 
10&
2&
Coussinet \\ \hline %& Cu Sn 12 P  \\ \hline 
11&
1&
Tube carter \\ \hline %& \\ \hline 
12&
1&
Bride de liaison droite \\ \hline %& Al Cu 4 Mg Si \\ \hline 
13&
1&
Carter cylindrique\\ \hline % & \\ \hline 
14&
1&
Axe excentré \\ \hline %& \\ \hline 
15&
4&
Vis à tête cylindrique à six pans creux\\ \hline % ISO 4762 - M5-50\\ \hline % & \\ \hline 
16&
1&
Chape mâle\\ \hline % & C 45 \\ \hline 
17&
2&
Goupille cylindrique \\ \hline %ISO 8734 - 2x16 \\ \hline %& \\ \hline 
18&
1&
Bielle rotule \\ \hline % &C 45 \\ \hline 
19&
1&
Cale de réglage \\ \hline %& \\ \hline 
20&
1&
Fermeture rotule \\ \hline %& \\ \hline 
21&
1&
Bielle à portée sphérique \\ \hline %& \\ \hline 
22&
3&
Vis à tête cylindrique à six pans creux  \\
&& ISO 4762 - M5-30 \\ \hline %& \\ \hline 
23&
1&
Goupille cylindrique ISO 8734 - 3x30  \\ \hline %& \\ \hline 
24&
1&
Chape femelle \\ \hline %& \\ \hline %C 45 \\ \hline 

25&
1&
Axe de chape \\ \hline % & \\ \hline 
26&
1&
Anneau élastique pour arbre, 4 x 0,4 \\ \hline % & \\ \hline 
27&
2&
Coussinet à collerette \\ \hline %& Cu Sn 12 P \\ \hline 
28&
1&
Bielle \\ \hline % & C 45 \\ \hline 
\end{tabular}
\end{center}

\begin{center}
\begin{tabular}{|l|l|l|}
\hline
Rep & Nb  & Désignation \\ \hline \hline % & Matière \\ \hline \hline
29&
3&
Vis à tête cylindrique à six pans creux \\ \hline % ISO 4762 - M5-18 \\ \hline % & \\ \hline 
30&
1&
Axe d'articulation \\ \hline % & \\ \hline 
31&
1&
Axe de sortie \\ \hline % & \\ \hline %100 Cr 6  \\ \hline 
32&
1&
Support d'axe de sortie \\ \hline %C 45 \\ \hline 
33&
1&
Ecrou hexagonal \\ \hline % ISO 4032 - M24  \\ \hline %& \\ \hline 
34&
1&
Rondelle plate \\ \hline % ISO 10673 – Type S - 24\\ \hline % &  \\ \hline 
35&
2&
Coussinet à collerette \\ \hline %& Cu Sn 12 P  \\ \hline 
36&
1&
Plateau support excentrique \\ \hline %& \\ \hline 
37&
1&
Vis à tête moletée \\ \hline %& \\ \hline 
38&
1&
Doigt de réglage \\ \hline % & C 22 \\ \hline 
39&
1&
Coussinet \\ \hline %& Cu Sn 12 P \\ \hline 
40&
1&
Entretoise  \\ \hline 
41&
2&
Anneau élastique pour arbre, 6 x 0,7  \\ \hline %& \\ \hline 
42&
1&
Anneau élastique pour alésage, 32 x 1,5 \\ \hline %& \\ \hline 
43&
1&
Anneau élastique pour arbre, 12 x 1 \\ \hline %& \\ \hline 
44&
1&
Arbre d’entrée \\ \hline %& \\ \hline 
45&
2&
Roulement à une rangée de \\ 
&& billes à contact radial \\ \hline %& \\ \hline 
46&
1&
Support roulements \\ \hline %& Al Cu 4 Mg Si \\ \hline 
47&
1&
Carter \\ \hline % & Al Cu 4 Mg Si \\ \hline 
48&
1&
Vis sans fin Z48 = 2 filets \\ \hline %& \\ \hline 
49&
2&
Boitier\\ \hline % & \\ \hline 
50&
2&
Roulement à une rangée de \\ 
&& billes à contact radial\\  \hline % & \\ \hline 
51&
2&
Joint à deux lèvres \\ \hline %& \\ \hline 
52&
1&
Arbre Creux\\ \hline % & \\ \hline
53&
1&
Vis à tête hexagonale ISO 4014-M6 \\ \hline %  & \\ \hline 
54 & 
1 &
Arbre \\ \hline % & \\ \hline 
55 &
1&
Roue dentée Z55= 60 dents  \\ \hline %&Cu Ni 2 Si  \\ \hline 
\end{tabular}
\end{center}

\end{multicols}

\fi

\question{Colorier le dessin de définition en utilisant la même couleur pour une même classe d'équivalence.}
\ifprof
\else
\fi

\question{Lister les classes classes d'équivalence.}
\ifprof
\else
\fi

\question{Donner le graphe de liaisons en précisant rigoureusement les liaisons. Justifier le choix des liaisons.}
\ifprof
\else
\fi


\question{Réaliser le schéma cinématique.}
\ifprof
\else
\fi


\ifprof
\else
\begin{center}
\includegraphics[width=\linewidth]{1019_01}
\end{center}
\fi


\ifprof
\else

\footnotesize
%\begin{center}
%\begin{tabular}{|p{.9\linewidth}|}
%\hline
%Indications (à vérifier...) :
%\begin{enumerate}
%\item $\vectv{B}{2}{0} = L\varphip(t)\vj{2} +\thetap(t)\left(L\vj{1}-R\vi{0}\right) $.
%\item  $\torseurcin{V}{2}{0} = \torseurl{\vecto{2}{0}=\left( \varphip(t)+\thetap(t) \right) \vk{0} }{ L\varphip(t)\vj{2} +\thetap(t)\left(L\vj{1}-R\vi{0}\right)}{B}$.
%\item $\vectg{B}{2}{0} =  L\varphipp(t)\vj{2}-L\varphip(t)\left(\varphip(t)+\thetap(t) \right)\vi{2}  + \thetapp(t)\left(L\vj{1}-R\vi{0}\right) - L\thetap^2(t)\vi{1}$.
%\end{enumerate} \\ \hline
%\end{tabular}
%\end{center}
\normalsize


\begin{flushright}
\footnotesize{Corrigé  voir \ref{B2:12:1018}.}
\end{flushright}%
\fi