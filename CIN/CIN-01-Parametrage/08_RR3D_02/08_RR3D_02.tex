\normalfalse \difficiletrue \tdifficilefalse
\correctionfalse

%\UPSTIidClasse{11} % 11 sup, 12 spé
%\newcommand{\UPSTIidClasse}{12}

\exer{Mouvement RR 3D  $\star$ \label{CIN:01:B2:13:PTSI:08}}
\setcounter{question}{0}\marginnote{\xpComp{CIN}{01}}%\UPSTIcompetence[2]{B2-13}
\index{Compétence B2-13-PTSI}
\index{Mécanisme à 2 rotations 3D}
\ifcorrection
\else
\marginnote{\textbf{Pas de corrigé pour cet exercice.}}
\fi

\ifprof
\else
Soit le mécanisme suivant. On a $\vect{AB}=H\vect{j_1}+R\vect{i_1}$ et $\vect{BC}=L\vect{i_2}$. On a $H=\SI{20}{mm}$, $r=\SI{5}{mm}$, $L=\SI{10}{mm}$. 
\begin{marginfigure}
\includegraphics[width=\linewidth]{08_RR3D_01}
\end{marginfigure}
\fi

\question{Réaliser le paramétrage du mécanisme.}
\ifprof
\else
\fi



\ifprof
\else
\footnotesize

\normalsize
\marginnote{Corrigé voir \ref{CIN:01:B2:13:PTSI:08}.}

\fi