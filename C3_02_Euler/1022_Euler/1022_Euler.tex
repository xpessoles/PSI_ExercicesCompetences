\normaltrue \difficilefalse \tdifficilefalse
\correctionfalse

%\UPSTIidClasse{11} % 11 sup, 12 spé
%\newcommand{\UPSTIidClasse}{11}

\exer{Schéma d'Euler$\star$ \label{C3:02:1022}}
\setcounter{question}{0}\UPSTIcompetence[2]{C3-02}
\index{Compétence C3-02}
\index{Schéma d'Euler explicite}
\ifcorrection
\else
\marginnote{\textbf{Pas de corrigé pour cet exercice.}}
\fi


\question{Donner la méthode de résolution numérique des équations différentielles suivantes en utilisant le schéma d'Euler explicite.}

%\begin{eqnarray}
%\right\{
$$
\begin{array}{l}
\ddot{\theta}(t) + \dfrac{g}{l}\sin \theta = 0\\
\theta(0) = 0 \quad \dot{\theta}(0) = 0 \\
\end{array} 
$$
%\right.
%\end{eqnarray}


\ifprof
On pose $y_0(t) = \theta(t) $ et $y_1(t) = \dot{\theta}(t) = y'_0(t) $. On a donc 
$$
\left\{
\begin{array}{l}
y'_0(t) = y_1(t) \\
y'_1(t) + \dfrac{g}{l} \sin y_0(t) = 0
\end{array} 
\right.
$$

Par ailleurs, $y_0(t) = 0 $ et $y_1(t) =0$.

En discrétisant, on a donc :
$$
\left\{
\begin{array}{l}
\dfrac{y_{0,k+1}-y_{0,k}}{h} = y_{1,k} \\
\dfrac{y_{1,k+1}-y_{1,k}}{h} + \dfrac{g}{l} \sin y_{0,k} = 0
\end{array} 
\right.
\Longleftrightarrow
\left\{
\begin{array}{l}
y_{0,k+1} = h y_{1,k} + y_{0,k}\\
y_{1,k+1} = - h \dfrac{g}{l} \sin y_{0,k}  + y_{1,k}
\end{array} 
\right.
$$
\else
\fi



 

\ifprof
\else
\begin{flushright}
\footnotesize{Corrigé  voir \ref{C3:02:1022}.}
\end{flushright}%
\fi