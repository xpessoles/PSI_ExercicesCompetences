\normaltrue \difficilefalse \tdifficilefalse
\correctionfalse

%\UPSTIidClasse{11} % 11 sup, 12 spé
%\newcommand{\UPSTIidClasse}{12}

\exer{Système bielle manivelle  $\star\star$ \label{C2:09:12}}
\setcounter{question}{0}\UPSTIcompetence[2]{C2-09}
\index{Compétence C2-09}
\index{TEC}
\index{Théorème de l'énergie cinétique}
\index{Bielle Manivelle}
\index{Moteur}
\ifcorrection
\else
\marginnote{\textbf{Pas de corrigé pour cet exercice.}}
\fi

\ifprof
\else
Soit le mécanisme suivant. On a $\vect{AB}=R\vect{i_1}$, $\vect{CB}=L\vect{i_2}$ et $\vect{AC}=\lambda(t) \vect{j_0}$. 
De plus, on note :
\begin{itemize}
\item $G_1 = A$ le centre d'inertie du solide \textbf{1}, $m_1$ sa masse et $\inertie{G_1}{1}=\matinertie{A_1}{B_1}{C_1}{0}{0}{0}{\rep{1}}$ sa matrice d'inertie;
\item $G_2$ le centre d'inertie du solide \textbf{2} tel que $\vect{CG_2}=\dfrac{L}{2} \vi{2}$, $m_2$ sa masse et $\inertie{G_2}{2}=\matinertie{A_2}{B_2}{C_2}{0}{0}{0}{\rep{2}}$ sa matrice d'inertie;
\item $G_3$ le centre d'inertie du solide \textbf{3} tel que $\vect{CG_3}=L_3 \vj{0}$, $m_3$ sa masse et $\inertie{G_3}{2}=\matinertie{A_3}{B_3}{C_3}{0}{0}{0}{\rep{3}}$ sa matrice d'inertie.
\end{itemize}
On note $C_m\vk{0}$ le couple moteur agissant sur le solide \textbf{1}, $F_h\vj{0}$ l'action du fluide sur \textbf{3}. L'accélération de la pesanteur est donnée par $\vect{g}=-g\vj{0}$.

\begin{center}
\includegraphics[width=\linewidth]{12_01}
\end{center}
\fi

On rappelle que la loi entrée sortie est donnée par la relation *** établie à l'exercice \ref{C2:06:12}.

\question{Tracer le graphe d'analyse en indiquant l'ensemble des actions mécaniques agissant sur les différents solides.}
\ifprof
\else
\fi

\question{Déterminer l'ensemble des puissances intérieures à l'ensemble \textbf{1+2+3}.}
\ifprof
\else
\fi

\question{Déterminer l'ensemble des puissances extérieures à l'ensemble \textbf{1+2+3}.}
\ifprof
\else
\fi

\question{Déterminer $\ec{1+2+3}{0}$.}
\ifprof
\else
\fi

\question{Déterminer la loi de mouvement en appliquant le théorème de l'énergie cinétique.}
\ifprof
\else
\fi



\ifprof
\else
\begin{flushright}
\footnotesize{Corrigé  voir \ref{C2:09:12}.}
\end{flushright}%
\fi