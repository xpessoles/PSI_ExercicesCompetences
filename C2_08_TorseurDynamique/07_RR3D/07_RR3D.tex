\normalfalse \difficiletrue \tdifficilefalse
\correctiontrue

%\UPSTIidClasse{11} % 11 sup, 12 spé
%\newcommand{\UPSTIidClasse}{12}

\exer{Mouvement RR 3D  $\star\star$ \label{C2:08:07}}
\setcounter{question}{0}\UPSTIcompetence[2]{C2-08}
\UPSTIcompetence[2]{C2-09}
\index{Compétence C2-08}
\index{Compétence C2-09}
\index{Torseur cinétique}
\index{Torseur dynamique}
\ifcorrection
\else
\marginnote{\textbf{Pas de corrigé pour cet exercice.}}
\fi

\ifprof
\else
Soit le mécanisme suivant. On a $\vect{AB}=R\vect{i_1}$ et $\vect{BC}=\ell\vect{i_2}+r\vect{j_2}$. On note $R+\ell=L = \SI{20}{mm}$ et $r=\SI{10}{mm}$. De plus :
\begin{itemize}
\item $G_1=B$ désigne le centre d'inertie de \textbf{1}, on note $m_1$ la masse de \textbf{1} et $\inertie{G_1}{1}=\matinertie{A_1}{B_1}{C_1}{0}{0}{0}{\bas{1}}$; 
\item $G_2$ désigne le centre d'inertie de \textbf{2} tel que  $\vect{BG_2}=\ell\vect{i_2}$, on note $m_2$ la masse de \textbf{2} et $\inertie{G_2}{2}=\matinertie{A_2}{B_2}{C_2}{0}{0}{0}{\bas{2}}$.
\end{itemize}
\begin{center}
\includegraphics[width=\linewidth]{07_RR3D_01}
\end{center}
\fi

\question{Exprimer le torseur dynamique $\torseurdyn{1}{0}$ en~$B$.}
\ifprof

Par définition, $\torseurdyn{1}{0} = \torseurl{\vectrd{1}{0}}{\vectmd{B}{1}{0}}{B}$.

\textbf{Calculons $\vectrd{1}{0}$}

$\vectrd{1}{0} = m_1\vectg{G_1}{1}{0}=m_1 \vectg{B}{1}{0} $

 \textbf{Calcul de $\vectv{B}{1}{0}$ : }  
$\vectv{B}{1}{0}=\deriv{\vect{AB}}{\rep{0}}$ 
$=\deriv{ R\vi{1}}{\rep{0}}$  
$= R\thetap \vj{1}$.


 \textbf{Calcul de $\vectg{B}{1}{0}$ : }  
$\vectv{B}{1}{0}=\deriv{\vectv{B}{1}{0}}{\rep{0}}$ 
$=\deriv{ R\thetap \vj{1}}{\rep{0}}$  
$=  R\thetapp \vj{1}-R\thetap^2 \vi{1}$.

Au final, $\vectrd{1}{0}= m_1\left(R\thetapp \vj{1}-R\thetap^2 \vi{1}\right)$.
\vspace{.5cm}

\textbf{Calculons $\vectmd{B}{1}{0}$}
$B$ est le centre d'inertie du solide 1; donc 
 d'une part, $\vectmd{B}{1}{0} = \deriv{\vectmc{B}{1}{0} }{\rep{0}}$.
 
 D'autre part, $\vectmc{B}{1}{0} = \inertie{B}{1}\vecto{1}{0}$
 $ =\matinertie{A_1}{B_1}{C_1}{0}{0}{0}{\bas{1}} \thetap \vk{0}$
  $ =C_1 \thetap \vk{0}$.
  
  Par suite, $\vectmd{B}{1}{0} = C_1 \thetapp \vk{0}$.
  
  Au final, 
$\torseurdyn{1}{0} = \torseurl{m_1\left(R\thetapp \vj{1}-R\thetap^2 \vi{1}\right)}{C_1 \thetapp \vk{0}}{B}$.


%$= \dfrac{\dd \vect{AB}}{\dd t}$
%$= \dfrac{\dd R\vi{1}}{\dd t}$
%$= R\thetap \vj{1}$.
%
%\noindent \textbf{Calcul de $\vectg{B}{1}{0}$ : }  $\vectg{B}{1}{0}= \dfrac{\dd $\vectv{B}{1}{0}$}{\dd t}$ 
%$= \dfrac{\dd \vect{AB}}{\dd t}$
%$= \dfrac{\dd}{\dd t}$
%$= R\thetap \vj{1}$


\else
\fi

\question{Déterminer $\vectmd{A}{1+2}{0}\cdot \vect{k_0}$}
\ifprof

Tout d'abord, $\vectmd{A}{1+2}{0} = \vectmd{A}{1}{0}+\vectmd{A}{2}{0}$.

\textbf{Calcul de $\vectmd{A}{1}{0} \cdot \vect{k_0}$ -- Méthode 1}

$\vectmd{A}{1}{0} \cdot \vect{k_0} = \left(\vectmd{B}{1}{0} + \vect{AB} \wedge \vectrd{1}{0} \right) \cdot \vect{k_0} $
$= \left(C_1 \thetapp \vk{0} + R\vi{1} \wedge m_1\left(R\thetapp \vj{1}-R\thetap^2 \vi{1}\right) \right) \cdot \vect{k_0} $
$= C_1 \thetapp  +  m_1R^2 \thetapp  $.


\textbf{Calcul de $\vectmd{A}{2}{0} \cdot \vect{k_0}$ -- Méthode 1}

$A$ est un point fixe. On a donc 
$\vectmd{A}{2}{0} \cdot \vect{k_0} = \deriv{\vectmc{A}{2}{0}}{\rep{0}}\cdot \vect{k_0} $
$= \deriv{\vectmc{A}{2}{0}\cdot \vect{k_0}}{\rep{0}} - \underbrace{\vectmc{A}{2}{0}\cdot \deriv{ \vect{k_0}}{\rep{0}}}_{\vect{0}}$.

$A$ est un point fixe. On a donc 
$\vectmc{A}{2}{0} \cdot \vect{k_0} = \left(\inertie{A}{2} \vecto{2}{0}\right)\cdot \vect{k_0} $

$\inertie{A}{2} = \inertie{G_2}{2}+\matinertie{0}{m_2R^2}{m_2R^2}{0}{0}{0}{\rep{2}}$ et
$\vecto{2}{0} = \thetap\vk{1}+\varphip\vi{2}=\thetap\left(\cos\varphi\vk{2}+\sin\varphi\vj{2}\right)+\varphip\vi{2}$. 

On a donc $\vectmc{A}{2}{0}= \matinertie{A_2}{B_2+m_2R^2}{C_2m_2R^2}{0}{0}{0}{\rep{2}}\begin{pmatrix}
\varphip \\
\thetap\sin\varphi \\
\thetap\cos\varphi
\end{pmatrix}_{\rep{2}}
= \begin{pmatrix}
A_2 \varphip \\
\thetap\sin\varphi \left(B_2+m_2R^2\right) \\
\thetap\cos\varphi\left(C_2+m_2R^2\right)
\end{pmatrix}_{\rep{2}}
$. 

De plus $\vk{1}=\cos\varphi\vk{2}+\sin\varphi\vj{2}$.
On a alors $\vectmc{A}{2}{0} \cdot \vect{k_0} = \thetap\sin^2\varphi \left(B_2+m_2R^2\right) +
\thetap\cos^2\varphi\left(C_2+m_2R^2\right)$.

Enfin, $\vectmd{A}{2}{0} \cdot \vect{k_0} = 
\left(B_2+m_2R^2\right)\left(\thetapp\sin^2\varphi  +
2 \thetap \varphip \cos\varphi\sin\varphi \right)+
\left(C_2+m_2R^2\right)\left(\thetapp\cos^2\varphi-
2\thetap\varphip\cos\varphi\sin\varphi\right)$.

\textbf{Conclusion}

$\vectmd{A}{1+2}{0}\cdot \vect{k_0} =  C_1 \thetapp  +  m_1R^2 \thetapp + \left(B_2+m_2R^2\right)\left(\thetapp\sin^2\varphi  +
2 \thetap \varphip \cos\varphi\sin\varphi \right)+
\left(C_2+m_2R^2\right)\left(\thetapp\cos^2\varphi-
2\thetap \varphip\cos\varphi\sin\varphi\right)$.
\else
\fi


\ifcolle
\question{Déterminer les lois de mouvements.}
%\question{Déterminer $\pext{2}{1}{0}$ et $\pext{1}{2}{0}$.}
\else
\fi


\ifprof
\else
\ifcolle
\else
\footnotesize
\begin{tabular}{|p{.95\linewidth}|}
\hline
\begin{enumerate}
\item $\torseurdyn{1}{0} = \torseurl{m_1\left(R\thetapp \vj{1}-R\thetap^2 \vi{1}\right)}{C_1 \thetapp \vk{0}}{B}$.
\item $\vectmd{A}{1+2}{0}\cdot \vect{k_0} =  C_1 \thetapp  +  m_1R^2 \thetapp + \left(B_2+m_2R^2\right)\left(\thetapp\sin^2\varphi  +
2 \thetap \varphip \cos\varphi\sin\varphi \right)+
\left(C_2+m_2R^2\right)\left(\thetapp\cos^2\varphi-
2\thetap \varphip\cos\varphi\sin\varphi\right)$.
\end{enumerate}\\
\hline
\end{tabular}
\normalsize
\fi
\begin{flushright}
\footnotesize{Corrigé  voir \ref{C2:08:07}.}
\end{flushright}%
\fi