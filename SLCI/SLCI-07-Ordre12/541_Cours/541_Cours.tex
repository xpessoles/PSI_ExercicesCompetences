\normaltrue \difficilefalse \tdifficilefalse
\correctiontrue

%\UPSTIidClasse{11} % 11 sup, 12 spé
%\newcommand{\UPSTIidClasse}{11}


\exer{Cours -- Systèmes d'ordre 1$\star$ \label{SLCI:07:541}}
\setcounter{question}{0}\marginnote{\xpComp{SLCI}{07}}%\UPSTIcompetence[2]{B2-06}
\index{Compétence SLCI-07}
\index{Identification}
\index{Identification temporelle}
\index{Ordre 1}
\index{Ordre 2}
\ifcorrection
\else
\marginnote{\textbf{Pas de corrigé pour cet exercice.}}
\fi


\question{Donner la forme canonique d'un système d'ordre 1.}

\question{Tracer la réponse indicielle pour un systèm de gain 0,5.}

\question{Tracer l'écart statique.}

\question{Donner le temps de réponse à 5\%.}

\question{Donner 2 méthodes d'identification des caratéristiques.}





%\ifprof
%\else
%\begin{solution}
%\begin{enumerate}
%\item  $H(p)=\dfrac{3,5}{1+8p}$.
%\item $H(p)=\dfrac{0,5}{1+\dfrac{2\times 0,1}{14,25}p+\dfrac{p^2}{14,25^2}}$.
%\item $H(p)=\dfrac{0,5}{1+\dfrac{2\times 0,3}{16}p+\dfrac{p^2}{16^2}}$.
%\end{enumerate}
%\end{solution}
%
\marginnote{Corrigé voir \ref{SLCI:07:541}.}
%
%\fi