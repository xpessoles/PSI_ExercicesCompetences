\normaltrue \difficilefalse \tdifficilefalse
\correctionfalse

%\UPSTIidClasse{11} % 11 sup, 12 spé
%\newcommand{\UPSTIidClasse}{12}
% ATS 2019
\exer{MCC à excitation indépendante $\star$ \label{B2:22:mcc:1023}}
\setcounter{question}{0}\UPSTIcompetence[2]{B2-22}
\index{Compétence B2-22}
\index{Moteur à courant continu}
\index{Moteur à courant continu}
\index{MCC}
\ifcorrection
\else
\marginnote{\textbf{Pas de corrigé pour cet exercice.}}
\fi


Une machine d'extraction est entraînée par un moteur à courant continu à excitation
indépendante.
L'inducteur est alimenté par une tension $u = \SI{600}{V}$ et parcouru par un courant d'excitation
d'intensité constante : $i = \SI{30}{A}$.
L'induit (rotor) de résistance $R = \SI{12}{m \Omega}$ est alimenté par une source fournissant une tension $U$
réglable de $\SI{0}{V}$ à sa valeur nominale : $U_N = \SI{600}{V}$.
L'intensité $I$ du courant dans l'induit a une valeur nominale : $I_N = \SI{1,50}{kA}$.
La fréquence de rotation nominale est $n_N = \SI{30}{tr/min}$. 

\section*{Démarrage}

\question{Réaliser un schéma de principe.}

\question{ En notant $\Omega$ la vitesse angulaire du rotor, la fem du moteur a pour expression : $E = K\Omega$ avec $\Omega$ en \si{rad/s}. Quelle est la valeur de $E$ à l'arrêt ($n = \SI{0}{tr/min}$) ? 
}
\ifprof
\begin{corrige}
\end{corrige}
\else
\fi

\question{Dessiner le modèle équivalent de l'induit de ce moteur en indiquant sur le schéma les
flèches associées à $U$ et $I$.}
\ifprof
\begin{corrige}
\end{corrige}
\else
\fi
\question{ Ecrire la relation entre $U$, $E$ et $I$ aux bornes de l'induit, en déduire la tension $U_d$ à
appliquer au démarrage pour que $I_d = 1,2 I_N$. }
\ifprof
\begin{corrige}
\end{corrige}
\else
\fi

\question{Citer un système de commande de la vitesse de ce moteur. }
\ifprof
\begin{corrige}
\end{corrige}
\else
\fi

\section*{ Fonctionnement nominal au cours d'une remontée en charge}

\question{Exprimer la puissance absorbée par l'induit du moteur et calculer sa valeur numérique.}
\ifprof
\begin{corrige}
\end{corrige}
\else
\fi

\question{Exprimer la puissance totale absorbée par le moteur et calculer sa valeur numérique. }
\ifprof
\begin{corrige}
\end{corrige}
\else
\fi

\question{Exprimer la puissance totale perdue par effet Joule et calculer sa valeur numérique. }
\ifprof
\begin{corrige}
\end{corrige}
\else
\fi

\question{Sachant que les autres pertes valent \SI{27}{kW}, exprimer et calculer la puissance utile et le
rendement du moteur.}
\ifprof
\begin{corrige}
\end{corrige}
\else
\fi

\question{Exprimer et calculer le couple utile $T_u$ et le couple
électromagnétique $T_{em}$.}
\ifprof
\begin{corrige}
\end{corrige}
\else
\fi


\ifprof
\else
\begin{flushright}
\footnotesize{Corrigé  voir \ref{B2:22:mcc:1023}.}
\end{flushright}%
\fi