\normaltrue \difficilefalse \tdifficilefalse
\correctionfalse
%\UPSTIidClasse{11} % 11 sup, 12 spé
%\newcommand{\UPSTIidClasse}{11}

\exer{Hublex $\star$ \label{C2:04:68}}
%% CCINP MP 2020
\setcounter{question}{0}\UPSTIcompetence[2]{C2-04}
\index{Compétence C2-04}
\index{Correcteur}
\index{Correcteur proportionnel intégral}
\index{Hublex}


\ifcorrection
\else
\marginnote{\textbf{Pas de corrigé pour cet exercice.}}
\fi



\ifprof
\else

L’architecture retenue pour contrôler le couple moteur est un asservissement en intensité, image du
couple moteur (voir équation précédente). Le schéma-blocs est représenté \autoref{fig_13}. Un convertisseur IU
fournit au calculateur une tension $\indice{u}{ic}(t)$ image de l’intensité de consigne $i_c(t)$, proportionnelle à cette
dernière de coefficient $\indice{K}{iu}$. De même, l’intensité réelle $i(t)$, mesurée par un capteur d’intensité de
coefficient $\indice{K}{capt}$, a pour image $\indice{u}{im}(t)$. L’écart, noté $\varepsilon(t) = \indice{u}{ic}(t) - \indice{u}{im}(t)$, est traité par le correcteur de fonction de transfert $C(p)$, qui impose la tension $u(t)$ aux bornes du moteur.
%On note $I_c(p)$, $\indice{U}{ic}(p)$, $\indice{U}{im}(p)$, $\varepsilon(p)$ les transformées de Laplace respectives de $i_c(t)$, $\indice{u}{ic}(t)$, $\indice{u}{im}(t)$ et $\varepsilon(t)$.

On donne la fonction de transfert du moteur : $H_m(p)=K_m\dfrac{1+\tau_m p}{1+\dfrac{2z_m}{\omega_{0m}}p+\dfrac{1}{\omega_{0m}^2}p^2}$.


\begin{figure}[H]
\centering
\includegraphics[width=\linewidth]{fig_13}
\caption{Schéma-blocs \label{fig_13}}
\end{figure}

\fi

\question{Préciser, en justifiant, quelle valeur donner à $\indice{K}{iu}$, caractéristique du convertisseur IU.}


\ifprof
\else
On prend, dans un premier temps, un correcteur purement proportionnel: $C(p)=K_p$.

On en déduit la fonction de transfert $H_I(p)=\dfrac{I(p)}{I_c(p)}$ :

$H_I(p)=\dfrac{K'}{1+K'}\dfrac{1+\tau_m p}{1+  
\dfrac{\dfrac{2z_m}{\omega_{0m}}+ K'\tau_m}{1+K'}p 
+ \dfrac{1}{\omega_{0m}^2(1+K')} p^2}$, avec $K'=\indice{K}{iu}K_pK_m$.

\fi

\question{Calculer l’expression littérale de l’erreur en régime permanent notée $\mu_s$, pour une entrée indicielle (i.e. $I_c(p)$ est un échelon unitaire), en fonction de $\indice{K}{iu}$, $\indice{K}{p}$ et $K_m$.}

La \autoref{fig_14} présente les diagrammes de Bode en boucle ouverte de l’asservissement étudié, en prenant $K_p=10$.


%Q30.
\question{Conclure, lorsque cela est possible, quant au respect des sousexigences de l’exigence «~1.7.1.1~» avec ce type de correcteur.}

Dans un deuxième temps, il est décidé d’utiliser un correcteur de type proportionnel intégral. Sa fonction de transfert est notée : $C(p)=K_p+\dfrac{K_i}{p}$.

%Q31.
\question{Préciser l’influence de cecorrecteur sur les performances du système. Justifier le choix de ce type de correcteur dans le cas étudié.}

On souhaite régler le correcteur afin de respecter les performances de précision et de stabilité.

%Q32.
\question{Tracer sur le DR4, les diagrammes de Bode asymptotique du correcteur, ainsi que l’allure des courbes réelles pour $K_p=10$ et $K_i=1000$. On précisera les valeurs numériques associées aux valeurs caractéristiques. On se propose de régler le correcteur grâce à la méthode suivante, en deux étapes :}
\textit{\begin{enumerate}
\item réglage de $K_p$ seul (c’est-à-dire en considérant $K_i=0$ tout d’abord), de façon à respecter les exigences de stabilité et de bande passante;
\item  réglage de $K_i$ de façon à éloigner la pulsation de cassure du correcteur à une décade vers la gauche de la pulsation de coupure à \SI{0}{dB}, de manière à ce que $\SI{0}{dB}$ ne soit quasiment pas modifiée.
\end{enumerate}}

%Q33.
\question{En suivant cette méthode, déterminer en justifiant la valeur numérique de $K_p$.}

%Q34.
\question{Déterminer alors la valeur numérique de $K_i$.}

 Une fois le correcteur réglé, on obtient les diagrammes de Bode en boucle ouverte (\autoref{fig_15}) et les réponses temporelles (\autoref{fig_16}), pour un échelon d’intensité $i_c(t)$ de \SI{2}{A}.

% Q35.
\question{Commenter le résultat obtenu vis-à-vis de l’exigence «~1.7.1.1.4~». Expliquer pourquoi, à partir des exigences du D6, cet asservissement n’est pas directement implanté en l’état dans le système.}

\ifprof
\else

Le correcteur reste inchangé. Afin de palier au problème identifié précédemment, on apporte une dernière évolution au sein du calculateur. Cela permet de respecter les exigences de l’asservissement. \autoref{fig_17} présente les réponses temporelles du système pour un échelon d’intensité $i_c(t)$ de \SI{2}{A}.
\fi

%Q36.
\question{Préciser quelle ultime modification a apporté le constructeur afin de respecter les exigences de l’asservissement.}


\ifprof
\else
\begin{figure}[H]
\centering
\includegraphics[width=\linewidth]{fig_14}
\caption{Diagrammes de Bode en boucle ouverte pour $K_p = 10$\label{fig_14}}
\end{figure}


\begin{figure}[H]
\centering
\includegraphics[width=\linewidth]{fig_15}
\caption{Diagrammes de Bode en boucle ouverte avec réglage du correcteur PI effectué \label{fig_15}}
\end{figure}

\begin{figure}[H]
\centering
\includegraphics[width=.95\linewidth]{fig_16}
\caption{Réponses temporelles avec réglage du correcteur PI effectué \label{fig_16}}
\end{figure}


\begin{figure}[H]
\centering
\includegraphics[width=.95\linewidth]{fig_17}
\caption{Réponses temporelles du système finalement implanté\label{fig_17}}
\end{figure}
\fi


\ifprof
\else

\noindent\footnotesize
 \fbox{\parbox{.9\linewidth}{
Éléments de corrigé : 
\begin{enumerate}
\item .
\end{enumerate}}}
\normalsize

\begin{flushright}
\footnotesize{Corrigé  voir \ref{C2:04:70}.}
\end{flushright}%
\fi