\normaltrue
\correctiontrue

%\UPSTIidClasse{11} % 11 sup, 12 spé
%\newcommand{\UPSTIidClasse}{11}

\exer{Mouvement R  $\star$ \label{C2:08:02}}
\setcounter{question}{0}\UPSTIcompetence[2]{B2-14}
\UPSTIcompetence[2]{B2-15}
\UPSTIcompetence[2]{C1-05}
\index{Compétence B2-14}
\index{Compétence B2-15}
\index{Compétence C1-05}
\index{Torseur des actions mécaniques transmissibles}
\index{Torseur d’une action mécanique extérieure}
\index{Principe fondamental de la statique}
\index{PFS}
\index{Mécanisme à 1 rotation}
\ifcorrection
\else
\marginnote{\textbf{Pas de corrigé pour cet exercice.}}
\fi

\ifprof
\else
Soit le mécanisme suivant. On a $\vect{AB}=R\vect{i_1}$ avec $R=\SI{20}{mm}$. La liaison pivot est motorisée par un moteur dont l'action mécanique sur \textbf{1} est donnée par $\vect{C_m}=C_m \vect{k_0}$.
On note $m_1$ la masse du solide 1 et $B$ son centre d'inertie. 
 La pesanteur est telle que $\vect{g}=-g\vect{j_0}$.
%On note $m_1$ la masse du solide 1, $B$ son centre d'inertie et $\inertie{G}{1}=\matinertie{A_1}{A_1}{A_1}{0}{0}{0}{\bas{1}}$.

\begin{center}
\includegraphics[width=\linewidth]{02_R_01}
\end{center}
\fi

\question{Réaliser le graphe d'analyse en faisant apparaître l'ensemble des actions mécaniques.}
\ifprof
\begin{center}
\includegraphics[width=.4\linewidth]{02_R_01_c}
\end{center}
\else
\fi

\question{Donner le torseur de chacune des actions mécaniques.}
\ifprof
$\torseurstat{F}{0}{1}=\torseurl{X_{01}\vi{1}+Y_{01}\vj{1}+Z_{01}\vk{1}}{L_{01}\vi{1}+M_{01}\vj{1}}{A}$,
$\torseurstat{F}{\text{pes}}{1}=\torseurl{-m_1g\vj{0}}{\vect{0}}{B}$, 
$\torseurstat{F}{\text{Mot}}{1}=\torseurl{\vect{0}}{C_m\vk{0}}{A}$.
\else
\fi


\question{Simplifier les torseurs dans l'hypothèse des problèmes plans.}
\ifprof
$\torseurstat{F}{0}{1}=\torseurl{X_{01}\vi{1}+Y_{01}\vj{1}}{\vect{0}}{A}$,
$\torseurstat{F}{\text{pes}}{1}=\torseurl{-m_1g\vj{0}}{\vect{0}}{B}$,
$\torseurstat{F}{\text{Mot}}{1}=\torseurl{\vect{0}}{C_m\vk{0}}{A}$.
\else
\fi

\question{Proposer une démarche permettant de déterminer l'effort que doit développer le moteur pour maintenir \textbf{1} en équilibre.}
\ifprof
On isole 1 et on réalise un théorème du moment statique en $A$ en projection sur $\vk{0}$.
\else
\fi



\ifprof
\else
\ifcolle
\else
\footnotesize
\begin{center}
\begin{tabular}{|p{.9\linewidth}|}
\hline
Indications :
\begin{enumerate}
\item .
\item $\torseurstat{F}{0}{1}=\torseurl{X_{01}\vi{1}+Y_{01}\vj{1}+Z_{01}\vk{1}}{L_{01}\vi{1}+M_{01}\vj{1}}{A}$,
$\torseurstat{F}{\text{pes}}{1}=\torseurl{-m_1g\vj{0}}{\vect{0}}{B}$,
$\torseurstat{F}{\text{Mot}}{1}=\torseurl{\vect{0}}{C_m\vk{0}}{A}$.
\item $\torseurstat{F}{0}{1}=\torseurl{X_{01}\vi{1}+Y_{01}\vj{1}}{\vect{0}}{A}$,
$\torseurstat{F}{\text{pes}}{1}=\torseurl{-m_1g\vj{0}}{\vect{0}}{B}$,
$\torseurstat{F}{\text{Mot}}{1}=\torseurl{\vect{0}}{C_m\vk{0}}{A}$.
\item TMS en $A$ en projection sur $\vk{0}$.
\end{enumerate} \\ \hline
\end{tabular}
\end{center}
\normalsize
\fi
\begin{flushright}
\footnotesize{Corrigé  voir \ref{C2:08:02}.}
\end{flushright}%
\fi