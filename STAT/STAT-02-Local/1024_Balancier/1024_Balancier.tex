\normaltrue \difficilefalse \tdifficilefalse
\correctionfalse

%\UPSTIidClasse{11} % 11 sup, 12 spé
%\newcommand{\UPSTIidClasse}{12}

\exer{Balancier du D2C $\star$ \label{STAT:02:B2:14:1024}}
\setcounter{question}{0}\marginnote{\xpComp{STAT}{02}}%\UPSTIcompetence[2]{B2-14}
\index{Compétence STAT-02}
\index{Balancier}
\index{D2C}
\index{Centre de gravité}
\index{Centre d'inertie}
\ifcorrection
\else
\marginnote{\textbf{Pas de corrigé pour cet exercice.}}
\fi
\ifprof
\else



Soit le balancier représenté sur la figure suivante. On considère que la masse volumique du matériau utilisé est constante. Par ailleurs l'accélération de la pesanteur est portée par $-\vect{y}$ : $\vect{g} = -g\vect{y}$.
\begin{marginfigure}
\centering
\includegraphics[width=\linewidth]{1024_02}
\end{marginfigure}
\fi



\question{Exprimer sous forme littérale l'expression de la position du centre d'inertie du solide.}
\ifprof
\else
\fi

\question{Déterminer $h$ pour que le centre d'inertie appartienne à l'axe de rotation $\axe{O}{x}$ du vilebrequin.}
\ifprof
\else
\fi

\question{Faire l'application numérique.}
\ifprof
\else
\fi


\question{Exprimer le torseur de pesanteur sur le vilebrequin en $G$ puis en $O$.}
\ifprof
\else
\fi



\ifprof
\else

\marginnote{Corrigé voir \ref{STAT:02:B2:14:1024}.}

\fi