\normaltrue \difficilefalse \tdifficilefalse
\correctionfalse

%\UPSTIidClasse{11} % 11 sup, 12 spé
%\newcommand{\UPSTIidClasse}{12}
% ATS 2019
\exer{Codeur incrémental $\star$ \label{SYS:01:538}}
\setcounter{question}{0}\marginnote{\xpComp{SYS}{01}}
\index{Compétence SYS-01}
\index{Caractériser un constituant de la chaîne d’information.}
\index{Capteurs}
\ifcorrection
\else
\marginnote{\textbf{Pas de corrigé pour cet exercice.}}
\fi

\begin{marginfigure}
\includegraphics[width=5cm]{538_01}
\end{marginfigure}

\question{Donner le rôle et le principe de fonctionnement (schémas) d'un codeur incrémental optique.}
\ifprof
\else
\fi

\question{Le codeur est équipé d'une voie de mesure et d'un disque à 25 fentes. Donner la résolution du capteur en degtés.}
\ifprof
\else
\fi

\question{Quelle sera la résolution du capteur s'il est équipé de deux voies de mesure ?}
\ifprof
\else
\fi

\ifprof
\else
Un codeur est monté en sortie d'un moteur. Le moteur est suivi d'un réducteur de rapport 100.

\fi

\question{Quelle est la résolution du capteur vis-à-vis de l'arbre de sortie du réducteur ?}
\ifprof
\else
\fi

\ifprof
\else
La position du codeur est transformée par un convertisseur numérique analogique en V. Ce convertisseur permet de convertir des angles variants de $-$10 tours à +10 tours sur une échelle de $-5$ à +\SI{5}{V}.   
\fi

\question{Donner le gain du convertisseur numérique analogique.}
\ifprof
\else
\fi


\ifprof
\else
\begin{flushright}
\footnotesize{Corrigé  voir \ref{SYS:01:538}.}
\end{flushright}%
\fi